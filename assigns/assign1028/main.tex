\documentclass[12pt]{article}

\usepackage{geometry}
\geometry{a4paper}

\usepackage{graphicx}
\usepackage{caption}

\usepackage{ctex}

\usepackage{amsmath}
\usepackage{amsfonts}
\usepackage{amsthm}
\usepackage{stmaryrd}

\usepackage{listings}
\usepackage{xcolor}

\definecolor{gray}{rgb}{0.5,0.5,0.5}

\lstset{
  frame=single,
  framerule=0pt,
  backgroundcolor=\color{gray!15},
  xleftmargin=2em,
}

\newcommand{\N}{\mathbb{N}}
\newcommand{\Tt}{\textrm{test1}}
\newcommand{\ttt}{\textrm{test0}}
\newcommand{\zz}{\mathbb{Z}_{2^{64}}}
\newcommand{\nn}{\{(n,m) | n,m \in \mathbb{N}, n \leq m\}}
\newcommand{\nw}{\{(n,\omega) | n \in \mathbb{N}\} \cup \{(n,\omega+1) | n \in \mathbb{N}\}}
\newcommand{\ww}{\{(\omega,\omega),(\omega,\omega+1),(\omega+1,\omega+1)\}}
\newcommand{\suc}{\textrm{succ}}
\newcommand{\lub}{\textrm{lub}}


\title{Assignment 1028}
\author{Ziqin Gong $\quad$ 520030910216}
\date{}

\begin{document}
  \maketitle

  \section*{1}

    \begin{gather*}
      \{(s,s) | (s,0) \in \llbracket x>0 \rrbracket\} \cup \\
      \{(s_0,s_1) | (s_0,n) \in \llbracket x>0 \rrbracket, n\in\zz\setminus\{0\},(s_1,0) \in \llbracket x>0 \rrbracket, \forall y\neq x,s_0(y)=s_1(y)\}
    \end{gather*}

  \section*{2}

  \subsection*{(1)}

    \begin{gather*}
      \{(s,s) | (s,0) \in \llbracket E \rrbracket\} \cup \\
      \{(s_0,s_1) | (s_0,n) \in \llbracket x>0 \rrbracket, n\in\zz\setminus\{0\}, (s_1,0)\in\llbracket E \rrbracket, \forall y\neq x,s_0(y)=s_1(y)\}
    \end{gather*}

  \subsection*{(2)}

    \begin{gather*}
      \llbracket C \rrbracket.\textrm{fin} \cup \{(s,s) | (s,0)\in\llbracket x\geq0 \rrbracket\}
    \end{gather*}

  \section*{3}

    \begin{proof}
      首先证明$(\mathbb{N},=)$是偏序集。

      (1) 自反性:由整数相等的自反性可得,$\forall n \in \mathbb{N}, n=n$。

      (2) 传递性:由整数相等的传递性可得,$\forall a,b,c \in \mathbb{N}, \textrm{若 }a=b,b=c, \textrm{则}a=c$。

      (3) 反对称性:易得$\forall a,b \in \mathbb{N}, \textrm{若} a=b \textrm{且} b=a, a=b$

      因此,$(\mathbb{N},=)$是偏序集。下面证明是完备的。

      任取$\mathbb{N}$中的一条链$S$,即$\forall a,b \in S, a=b \textrm{ or } b=a$,即$S$中所有元素都相等,那么其上确界也就是该元素的值。
      因此$\mathbb{N}$中的每一条链都有其上确界,故$(\mathbb{N},=)$是完备偏序集。
    \end{proof}

  \section*{4}

    \begin{proof}
      首先证明$(A,\subseteq)$是偏序集。

      (1) 自反性:由集合包含的自反性可得,$\forall S \in A, S \subseteq S$。

      (2) 传递性:对任意$S_1,S_2,S_3 \in A$,若$S_1 \subseteq S_2, S_2 \subseteq S_3$,则由集合包含的传递性可得,$S_1 \subseteq S_2$。

      (3) 反对称性:对任意$S_1,S_2 \in A$,若$S_1 \subseteq S_2, S_2 \subseteq S_1$,则由集合包含的反对称性可得,$S_1 = S_2$。

      因此,$(A,\subseteq)$是偏序集。下面证明是完备的。

      任取$A$中的一条链$C$,存在$U=\bigcup_{S_i \in C}S_i$为$C$的上确界。易证其满足$\textrm{lub}(C)$的两条要求。
      下面证明$U$是自然数集$\mathbb{N}$的有穷子集。

      显然$U$中的所有元素都是自然数,故$U\subseteq\mathbb{N}$。
      若$U$是无穷集合,由于$S_i$都是有穷集合,故$C$应是无穷集合。
      又因为$C$中任意两个元素都有包含关系,不妨设$S_1 \subseteq S_2 \subseteq S_3 \subseteq \cdots$,
      则该无穷长链的最后的元素$S_\infty$必包含无限多的元素,即无穷集合,与$S_i$为有穷集合矛盾。
      故$U$是有穷集合。综上,$U$是自然数集的有穷子集,即$U \in A$。

      因此,$(A,\subseteq)$是完备偏序集。
    \end{proof}

  \section*{5}

    \begin{proof}
      当$m=0$时,$F(n)=n$,显然在$(\mathbb{N},D)$上单调。

      当$m>0$时,任取$a,b \in \mathbb{N}$使得存在$p\in\mathbb{N}\text{ s.t. }b=pa\text{ i.e. }a|b$。
      则
      \begin{gather*}
        F(b) = F(pa) = \textrm{gcd}(pa,m) = \textrm{gcd}(p,m) \cdot \textrm{gcd}(a,m) = qF(a),
      \end{gather*}
      其中$q=\textrm{gcd}(p,m)\in\mathbb{N}$,
      即$F(a) | F(b)$,故$F(n)$在$(\mathbb{N},D)$上单调。
    \end{proof}

  \section*{6}

    \begin{proof}
      首先证明单调性。
      当$m=0$时,$F(n)=0$,显然在$(\mathbb{N},D)$上单调。

      当$m>0$时,若$n=0$,$F(0)=0$。由于任意$n\in\mathbb{N}_+$,$n | 0$且$F(n) | 0$,故满足单调性。

      若$n \in \mathbb{N}_+$,
      任取$a,b \in \mathbb{N}_+$使得存在$p\in\mathbb{N}_+\text{ s.t. }b=pa\text{ i.e. }a|b$。
      则
      \begin{gather*}
        F(b) = F(pa) = \textrm{lcm}(pa,m) = \frac{p}{\textrm{lcm}(p,m)} \cdot \textrm{lcm}(a,m) = qF(a),
      \end{gather*}
      其中$q=p/\textrm{lcm}(p,m)\in\mathbb{N}_+$,
      即$F(a) | F(b)$,故$F(n)$在$(\mathbb{N},D)$上单调。下面证明连续性。

      任取$\N$上的一条链$C$。当$m=0$时,$F(n)=0,\forall n\in\N$,故显然$F(\lub(C))=\lub(F(C))=0$。

      当$m>0$时,若$C$是有穷集合,不妨设$c_1 | c_2, c_2 | c_3, \cdots, c_{n-1} | c_n$,则$\lub(C)=c_n$。
      由单调性可得$F(c_i) | F(c_{i+1})$,故$\lub(F(C))=F(c_n)=F(\lub(C))$。

      若$C$是无穷集合,则$F(C)$也是无穷集合,二者都有共同的上确界$0$,而$F(0)=0$,故也满足$\lub(F(C))=F(\lub(C))$。

      综上,$F(n)$是$(\N,D)$的单调连续函数。
    \end{proof}

  \section*{7}

    \begin{proof}

      \textbf{(1)} 首先证明$(A,\leq_A)$是偏序集。

        I. 自反性:任取$n \in A$,若$n\in\N$,则$(n,n) \in \nn$;若$n=\omega$或$\omega+1$,则$(n,n) \in \ww$。故
        \begin{gather*}
          \forall n \in A, (n,n) \in \leq_A。
        \end{gather*}

        II. 传递性:任取$l,m,n \in A$。若$l,m,n \in \N$,由$\leq$的传递性易得;

        若$(l,m) \in \nn, (m,n) \in \nw$,则$(l,n)\in\nw$;

        若$(l,m)\in\nw,(m,n)\in\ww$,则$(l,n)\in\nw$;

        若$(l,m)\in\ww,(m,n)\in\ww$,则$(l,n)\in\ww$。

        故
        \begin{gather*}
          \forall l,m,n \in A,\text{ if }(l,m)\in\leq_A,(m,n)\in\leq_A,\text{ then }(l,n)\in\leq_A。
        \end{gather*}

        III. 反对称性:任取$n,m\in A$,若$n,m\in\N$,有$n\leq m,m \leq n$,由$\leq$的反对称性得$n=m$;

        若$n\in\N,m\in\{\omega,\omega+1\}$,显然不存在,舍去;

        若$n,m\in\{\omega,\omega+1\}$,则
        \begin{gather*}
          (n,m),(m,n)\in\ww\text{ iff }n=m=\omega\text{ or }n=m=\omega+1。
        \end{gather*}

        故
        \begin{gather*}
          \forall n,m \in A,\text{ if }(n,m)\in\leq_A,(m,n)\in\leq_A,\text{ then }n=m。
        \end{gather*}

        综上,$(A,\leq_A)$是偏序集。下面证明其完备。

        任取$A$中的一条链$C$,若$C\subseteq\N$,显然$\lub(C)=\max C$;

        若$\omega \in C, \omega+1 \notin C$,由$\leq_A$的定义可得$\lub(C)=\omega$;

        若$\omega+1\in C$,由$\leq_A$的定义可得$\lub(C)=\omega+1$。

        故任意一条链都有其上确界。综上,$(A,\leq_A)$是完备偏序集。

      \textbf{(2)} 任取$n,m\in A$满足$(n,m)\in\leq_A$。

        若$n,m\in\N$,由$n\leq m$易得$\suc(n)\leq\suc(m)$;

        若$n\in\N,m\in\{\omega,\omega+1\}$,$\suc(n)\in\N$,$\suc(m)=\omega+1$,显然$(\suc(n),\suc(m))\in\leq_A$;

        若$n,m\in\{\omega,\omega+1\}$,$\suc(n)=\suc(m)=\omega+1$,即$(\suc(n),\suc(m))\in\leq_A$。

        综上,$\suc$是单调函数。下面举反例说明其不连续。

        取$\N$作为$A$上的一条链$C$,则$\suc(C)=\N_+,\lub(C)=\omega$,而$\lub(\N_+)=\omega$,$\suc(\lub(C))=\omega+1$,二者不等。
        故$\suc$不连续。

      \textbf{(3)} 易得$\bot=0$,则
        \begin{gather*}
          \lub(\bot,\suc(\bot),\suc(\suc(\bot)),\cdots)=\lub(0,1,2,\cdots)=\omega。
        \end{gather*}
        而$\suc(\omega)=\omega+1\neq\omega$,故$\lub(\bot,\suc(\bot),\suc(\suc(\bot)),\cdots)$不是$\suc$的不动点。
    \end{proof}

\end{document}
