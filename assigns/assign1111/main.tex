\documentclass[11pt]{article}

\usepackage{geometry}
\geometry{a4paper}

\usepackage{graphicx}
\usepackage{caption}

\usepackage{ctex}

\usepackage{amsmath}
\usepackage{amsfonts}
\usepackage{amsthm}
\usepackage{stmaryrd}

\usepackage{listings}
\usepackage{xcolor}

\definecolor{gray}{rgb}{0.5,0.5,0.5}

\lstset{
  basicstyle=\footnotesize\ttfamily,
  frame=single,
  framerule=0pt,
  backgroundcolor=\color{gray!15},
  xleftmargin=2em,
}


\title{Assignment 1111}
\author{Ziqin Gong $\quad$ 520030910216}
\date{}

\begin{document}
  \maketitle

  \section*{1}

    \begin{lstlisting}
exists x'. x == 0 && 100 <= x' + y + z && x' <= 0
    \end{lstlisting}

  \section*{2}

    \begin{lstlisting}
exists x'. x == x' + y && 0 <= x' + y <= 100 && x' * y <= 100
    \end{lstlisting}

  \section*{3}

    \begin{lstlisting}
exists x' y'. y == x - y' && x' == m && y' == n && x == x' + n
    \end{lstlisting}

  \section*{4}

    \subsection*{(1)}

      \begin{lstlisting}
exists v. store(x, u + 1) * store(x + 8, v) * listrep(v)
      \end{lstlisting}

    \subsection*{(2)}

      记$P=\ $\texttt{listrep(x)},$e=\ $\texttt{x != 0 \&\& * x <= 0},
      $Q=\ $\texttt{listrep(x)}。则
      \begin{lstlisting}
{P && e} ==
{exists u v. store(x, u) * store(x + 8, v) * listrep(v) &&
             x != 0 && u <= 0}
* x = * x + 1
{exists u v. store(x, u+1) * store(x + 8, v) * listrep(v) &&
             x != 0 && u <= 0} which implies
{exists u v. store(x, u) * store(x + 8, v) * listrep(v)}
== {listrep(x)}
      \end{lstlisting}
      \begin{lstlisting}
{P && !e} ==
{exists u v. store(x, u) * store(x + 8, v) * listrep(v) &&
             (x == 0 || * x > 0)}
x = x
{exists u v. store(x, u) * store(x + 8, v) * listrep(v) &&
             (x == 0 || * x > 0)} which implies
{listrep(x)}
      \end{lstlisting}

      因此,由Consequence Rule可得:

      \begin{center}
        $\{\ P\ \&\&\ e\ \}\ $\texttt{* x = * x + 1}$\ \{\ Q\ \}$

        $\{\ P\ \&\&\ !e\ \}\ $\texttt{x = x}$\ \{\ Q\ \}$
      \end{center}

      故原霍尔三元组成立。

    \subsection*{(3)}

      记$P=Q=\ $\texttt{listrep(x) * listrep(y)}, $e=\ $\texttt{x != 0}。
      \begin{lstlisting}
{P && e} ==
{exists u v. x != 0 && store(x, u) * store(x + 8, v) * listrep(v) *
             listrep(y)}
t = x
{exists u v. t == x && x != 0 && store(x, u) * store(x + 8, v) *
             listrep(v) * listrep(y)}
x = * (x + 8)
{exists u v x'. x == v && t == x' && x' != 0 && store(x', u) *
                store(x' + 8, v) * listrep(v) * listrep(y)}
化简:
{exists u v. x == v && t != 0 && store(t, u) * store(t + 8, v) *
             listrep(v) * listrep(y)}
* (t + 8) = y
{exists u v. x == v && t != 0 && store(t, u) * store(t + 8, y) *
             listrep(v) * listrep(y)}
y = t
{exists u v y'. y == t && x == v && t != 0 && store(t, u) *
                store(t + 8, y') * listrep(v) * listrep(y')}
化简:
{exists u y'. y != 0 && store(y, u) * store(y + 8, y') * listrep(y') *
              listrep(x)}
== {listrep(x) * listrep(y)} == { Q }
      \end{lstlisting}
      \begin{lstlisting}
{P && !e} == {x == 0 && listrep(y)}
x = x
{x == 0 && listrep(y)} == { Q }
      \end{lstlisting}

      因此,由Consequence Rule可得:

      \begin{center}
        $\{\ P\ \&\&\ e\ \}\ $\texttt{t = x; x = * (x + 8); * (t + 8) = y; y = t}$\ \{\ Q\ \}$

        $\{\ P\ \&\&\ !e\ \}\ $\texttt{x = x}$\ \{\ Q\ \}$
      \end{center}

      故原霍尔三元组成立。

\end{document}
