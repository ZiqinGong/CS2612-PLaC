\documentclass[11pt]{article}

\usepackage{geometry}
\geometry{a4paper}

\usepackage{graphicx}
\usepackage{caption}

\usepackage{ctex}

\usepackage{amsmath}
\usepackage{amsfonts}
\usepackage{amsthm}
\usepackage{stmaryrd}

\usepackage{listings}
\usepackage{xcolor}

\definecolor{gray}{rgb}{0.5,0.5,0.5}

\lstset{
  basicstyle=\footnotesize\ttfamily,
  frame=single,
  framerule=0pt,
  backgroundcolor=\color{gray!15},
  xleftmargin=2em,
}

\newcommand{\iv}{\texttt{IV}}


\title{Assignment 1115}
\author{Ziqin Gong $\quad$ 520030910216}
\date{}

\begin{document}
  \maketitle

  \section*{1}

    记\texttt{IV}=\texttt{listrep(x)}。首先证明\iv 为循环不变量。显然\iv 可以推出
    \texttt{x}求值安全。那么

    \begin{lstlisting}
{ IV && x } ===
{ exists x'. x == x' && listrep(x') && x' != 0 } ===
{ exists x' u v. x == x' && x' != 0 && store(x', u) * store(x' + 8, v) *
                                       listrep(v) }
x = * (x + 8)
{ exists x' u v. x == v && x' != 0 && store(x', u) * store(x' + 8, v) *
                                      listrep(v) } which implies
{ exists v. x == v && listrep(v) } ===
{ listrep(x) } === { IV }
    \end{lstlisting}

    因此\iv 是循环不变量。注意到前条件可推出\iv,且\texttt{IV \&\& x == 0}可以推出
    后条件,故由Consequence Rule可证该三元组成立。

  \section*{2}

    记\iv=\texttt{array0(x, i) * array(x + 8 * i, n - i) \&\& 0 <= i <= n}。显然
    \iv 保证求值安全。

    \begin{lstlisting}
{ IV && i < n } ===
{ exists x' i' u. x == x' && i == i' && 0 <= i' < n &&
                  [i' == 0 || store(x', 0) * array0(x' + 8, i' - 1)] &&
                  store(x' + 8 * i', u) * array(x' + 8 * (i'+1), n-i'-1) }
* (x + i * 8 ) = 0
{ exists x' i' u. x == x' && i == i' && 0 <= i' < n &&
                  [i' == 0 || store(x', 0) * array0(x' + 8, i' - 1)] &&
                  store(x' + 8 * i', 0) * array(x' + 8 * (i'+1), n-i'-1) }
===
{ exists x' i' u. x == x' && i == i' && 0 <= i' < n &&
                  [i' == 0 || store(x', 0) * array0(x' + 8, i')] *
                  array(x' + 8 * (i' + 1), n - i' - 1) }
i = i + 1
{ exists x' i' u. x == x' && i == i' + 1 && 0 <= i' < n &&
                  [i' == 0 || store(x', 0) * array0(x' + 8, i')] *
                  array(x' + 8 * (i' + 1), n - i' - 1) }
===
{ exists x' i' u. x == x' && i == i' && 0 < i' <= n &&
                  store(x', 0) * array0(x' + 8, i' - 1) *
                  array(x' + 8 * i', n - i') }
===
{ array0(x, i) * array(x + 8 * i, n - i) && 0 < i <= n } which implies
{ IV }
    \end{lstlisting}

    因此\iv 是循环不变量。注意到前条件可推出\iv,且\texttt{IV \&\& i >= n}可推出
    后条件,故由Consequence Rule可证该三元组成立。

\end{document}
